\documentclass{IEEEtran}
\usepackage[utf8]{inputenc}

\title{CS 6000 Journal 6}
\author{David Stout}
\date{October 2018}


\begin{document}

\maketitle
\begin{abstract}
    Journal 6 entry talks about the learning process that I have come through so far, and the help that I have found when others have constructively criticized my rough draft.
\end{abstract}


\begin{quote}
"For the things we have to learn before we can do them, we learn by doing them." - Aristotle
\end{quote}
\section{Introduction}
For this week I have attempted to make a more cohesive story. To do this I have had multiple people read my rough
draft and outline, in the attempt to get some constructive feedback. I have had both people in the field and 
people that have no background in Computer Science, so that I may make it readable for both my peers and anyone 
that is interested. 

I spoke at length with my PhD advisor about how to properly construct the story that I am attempting to tell. I 
found that I was very apprehensive about whether I was on the right path, and after talking with Dr. Zhuang and 
getting my feedback from Dr. Boult, I am beginning to feel a little more comfortable with what I have done so 
far. I know that there is a lot of work ahead, but I feel that I have some solid help when I need it. I also 
spoke with other students in class and we agreed to help each other with editing over the next few days, and I 
believe that by reading and editing each others papers we will all be able to put together papers that will be 
much better than if we were just flying blind. 

Normally, I am one that really just doesn't care for outlines. I tend to be the person that writes a paper first 
and does the outline last if one is required, but I have found that with this survey paper it is best to have an 
idea of how I want the paper to progress. With that in mind, I have spent a good deal of time on an outline. I 
have found that the outline has helped to give me some direction and that some of my sources just don't fit in 
the story that I am trying to tell. 

Since I am lacking on sources anyways, I have been researching more sources that help to tell my story based on 
my outline. I have found that doing this is helping to sharpen my focus and clarify the story that I am trying to
tell. Not only the details of what I am looking at, but why it is important to new people coming into the field 
and how it can potentially impact in the long run.

This weekend has been a lot of writing and determining what will work and what needs to go. But as Dr. Boult had 
said I am getting into the habit of not deleting what I have written and just moving it to another document. I 
find that after a little time, I am able to go back at what I thought would not work for this paper and reword it
to fit into an area that I need. It has really helped to have some partial thoughts down on "paper" that I am 
able to expand upon, especially when I feel stuck. Sometimes I may not even use the "stubs" that I am looking at,
but they make me think about the concept in a different way and help to alleviate the writer's block temporarily.

Over the next week I plan to continue to revise and expand on my story, while asking anyone that I can for 
feedback. I have found that the more eyes that are on the "problem", the better my editing has gone. Others have 
seen things that I never would have thought of and after making the changes, I look at the revision and think 
that it should have been that way the entire time. While this project may not be a group effort, I am very 
thankful for the assistance that others have been able to provide. 
\end{document}