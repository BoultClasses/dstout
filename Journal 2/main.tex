\documentclass{IEEEtran}
%\usepackage[utf8]{inputenc}
\usepackage{cite}
\title{CS 6000 Journal 2}
\author{David Stout}
\date{September 7, 2018}

%\usepackage{natbib}
\usepackage{graphicx}

\begin{document}

\maketitle

\section{Process and Learning}
\subsection{The Browse}
For The Browse, I found that it was a little difficult to not read too much. I began my first few papers 
and found myself trying to read parts of the intro and conclusion to learn more about the paper to 
determine if I would want to scan it later, but after a few papers I realized that this is not the method 
that we are going for. So I tried to just do as the class slides suggested and read the abstract and scan 
the figures, then to mainly go with my gut if I was interested in the paper. I found this method to be much 
faster and really helped to weed out the papers that I just didn't really care about currently. It also 
brought to light papers that I may not care about now, but might want to read later for future research, 
which was nice to put in my notes in Zotero so that I can quickly reference them again at a later date.
\subsection{The Scan}
The scan/skim portion of reading was something that I believe came a little more naturally. I have always 
attempted to get an overall picture of what I am reading by quickly scanning each paragraph and looking for 
the key ideas, and really focusing on those ideas as well as the conclusion. I found that the timing 
recommended was just about perfect to be able to answer the six C's and the three main questions. 
\subsection{Critical read}
The Critical read portion of this assignment will take me some further practice and knowledge. I believe that the 
process is sound and was able to glean a lot of information from each paper, but my current level of knowledge is not 
up to the task for all papers. Practice and further study is the only thing that will help in this area, as my 
knowledge grows I believe that I will be better equipped to answer all the critical questions that I need to.

\section{Papers Read}
\subsection{\underline{Measuring Neural Efficiency of Program Comprehension}\cite{siegmund_measuring_2017}}
Raw Notes:
\\

Is the problem carefully stated/formulated?

-update theories of how programmers understand programs for more modern cognitive neuro-science methods
\\

Is it a meaningful/real problem?

-the current theories rely on decades old information, neuro-science has been greatly improved upon since
\\

Is it the RIGHT problem (formulation vs description)?

-Problem formulation is small in scope and could use expansion to bring significant results
\\

Can you make a list of assumptions, explicit and more importantly implicit about the problem?

-data and theories outdated

-science has improved thus needing more research

-newer measurement tech allows for better analysis

-fMRI is best tool to measure the brain activation (over EEG, PET, etc.)
\\

What well known problems are related or the same? 

-foreign language comprehension might use a semantic/bottom-up approach

-finding confusing code patterns in source code, likely the same semantic/bottom-up comprehension approach
\\

Are there simple solutions the authors do not seem to have considered?

-other tools exist for measuring brain activity and locality
\\

What are the limitations of the solution (including limitations the authors might not have noticed or clearly 
admitted)?

-language comprehension and approach to cognition may be affected by cultural upbringing

-participants were shown small snippets (20 lines) where larger snippets could have a negative effect on bottom-up 
comprehension, due to user fatigue, over semantic cues
\\

Is the logic of the paper clear and justifiable, given the assumptions, or is there a flaw in the reasoning?

-logic seems sound but the body of work is based entirely off of main author's previous work so there isn't much to 
compare and contrast with
\\

Do all the pieces of their work fit together logically?

-logical flow is sound

If the authors present data, did they gather the right data to substantiate their argument, and did they appear to 
gather it in the correct manner?

-Specialized data gathering techniques require knowledge of fMRI and software specialized to imaging, but appears 
sound (to observer unfamiliar with radiography)
\\

Did they have enough data to make a statistically sound decision?

-only 11 participants

-seems like participants were conditioned to snippets and then shown only 12 similar snippets each, possibly leading 
the data where the researchers wanted it to go due to participant training

-data measurement conditions were only taken into account regarding time not other parameters, ie. how the 
participants would react to the confines of the fMRI and only being able to see the snippets from a mirror mounting.
\\

Did they interpret the data in a reasonable manner?

- data seems to have a reasonable interpretation, scans show elevated activity in the areas of the brain similiar to 
previous study

-programming plans and language comprehension may be in the same neural circuit locations
\\

Would other data be more compelling?

-a comparison of their fMRI data with EEG or PET data from the same participants
\\

Did they compare with actual state of the art performance?

-comparison was done versus main author's previous body of work

-atoms of confusion team did something a little similar using EEG machine, no comparison done
\\

Was the same data used for training and testing?

-very similar to condition participants for quickness, but not the same data set
\\

What were the results? Did they do what they set out to do?

-results show semantic-cues to have a better neuro-cognition result for program comprehension over bottom-up 

-layout and beacons did not significantly affect comprehension (seemingly odd)
\\

On what dimension(s) did they advance the art?

-Semantic Chunking theory was supported by their evidence

-Neural Efficiency was shown to be better in expert-level participants over novice-level 
\\

Have the authors been cutting corners (intentionally or unintentionally)?

-mention plenty of studies that utilize different measurement methods, but did not compare/contrast against any but 
the main author's previous study
\\

Would results be reproducible? 

-results were reproducible to main author's initial study, possible to reproduce results for this study but takes a 
specialization in fMRI setup and analysis
\\

Are the claims modest enough?

-claims that no evidence to support beacons and program layout affect comprehension tempered with idea of effect too 
small to be measured with experiment setup
\\

Need further information on:

-BrainVoyagerQX software to process data

-Talairach brain?

-Mention study conducted by Floyd et al using fMRI but no real comparisons drawn

\subsection{\underline{Understanding Misunderstandings in Source Code}\cite{gopstein_understanding_2017}}

Raw notes: 

Is the problem carefully stated/formulated?

-programmer intent conveyed to people is often subject to interpretation and misinterpretations confusing code patterns lead to bugs in code
\\

Is it a meaningful/real problem?

-atoms can create bugs in production code

-create faulty products, higher costs, and diminished productivity that is avoidable

-problem has real-world applications and consequences
\\

Is it the RIGHT problem (formulation vs description)?

-yes, the description of the problem is tempered with real-world examples
\\

Can you make a list of assumptions, explicit and more importantly implicit about the problem?

-syntactic and semantic level bugs cost companies money and time, and are avoidable if the atoms are identifiable
\\

What well known problems are related or the same?

-Program comprehension using beacons, layouts, and plans 

-Cloudbleed

-Apple "goto fail" SSL bug

-Language comprehension 

-IOCCC (International Obfuscated C Code Contest) - shows inherently confusing C code exists and is a problem
\\

Are there simple solutions the authors do not seem to have considered?

-attacking the problem with the simplest of solutions, using the alternative non-confusing code to accomplish the same
tasks as confusing code

-more complex solutions seem to be the advent of dynamic languages, bring their own issues
\\

Limitations of the solution?

-confusing code may not be cross cultural, what is confusing to americans may not be to other cultures and vice versa

-solution only works if standards and syntaxes can be updated and most programmers get on board

-no automated tool to find and replace atoms in production code
\\

Is the logic of the paper clear and justifiable, given the assumptions, or is there a flaw in the reasoning?

-logic is clear, examples of code bugs in production code that use identified atoms lend justification
\\

Do all the pieces of their work fit together logically?

-work flows together, with most work supporting each other
\\

If the authors present data, did they gather the right data to substantiate their argument, and did they appear to 
gather it in the correct manner?

-data was gathered between two models, confusing code and equivalently transformed non-confusing code

-appears to have been gathered in single-blind testing with appropriate power
\\

Did they have enough data to make a statistically sound decision?

-the two experiments were done with an appropriate power analysis done to ensure the data was statistically sound

- experiments were randomized single-blind  
\\

Did they interpret the data in a reasonable manner?

-data was adjusted for homogeneity and corrected for correlated data

-statistical analysis using R package to report chi-squared stats deriving a p-value seems to be correct in method as 
far as my knowledge of chi-squared can determine

-effect size was determined using standard statistic analysis
\\

Would other data be more compelling? 

-recruiting participants that are currently producing code may give better insight - participant level of expertise 
not discussed

-participants gave a self perceived notion of their C/C++ knowledge, contains bias
\\

Was the same data used for training and testing?

-simple snippets used for training

-transformed IOCCC submissions used for testing
\\

What were the results? Did they do what they set out to do?

-results shown indicated that participants performed marginally better on clarified code

-clarified code was less confusing to participants and 15/19 tested atom candidates showed a significant p-value
\\

On what dimension(s) did they advance the art?

-identifying atoms is the beginning but implementation of the finding will truly advance

-changing style guides and syntax handling is necessary

-larger code project experiment showed that atoms had significant impact on correctness, seemingly in a compounding 
effect
\\

Would results be reproducable? 

-experiments are done in a manner that is reproducable and all tools are provided to the researcher
\\

Problematic experimental setup? 

-participants were asked their perceived level of competency, not tested to find competency

-atoms were identified soley in C language, possible to test other languages for cross language atoms
\\

Methodological misunderstanding? 

-methodology is perceived as being cross platform, but may not be for certain languages
\\

Do the numbers add up? 

-the statistical analysis provided seems sound and fairly implemented
\\

Are the generalizations valid? 

-code patterns are generalized as being cross language issues when certain atoms are inherent to C, the premise of 
confusing code is definitely something that crosses languages, but the patterns themselves will vary from language to 
language most likely
\\

Are the claims modest enough?
-threats to validity are sound, and the researchers seem to acknowledge that much more research is necessary for truly
impacting results

-a good starting point is provided in the paper for future research

\clearpage
\section{Conference Proceedings}
E. Bodden and W. Schafer, Eds., 11TH JOINT MEETING OF THE EUROPEAN SOFTWARE ENGINEERING CONFERENCE AND THE ACM SIGSOFT SYMPOSIUM ON THE FOUNDATIONS OF SOFTWARE ENGINEERING, September 4-8 2017, Paderborn, Germany

\nocite{*}

\bibliographystyle{ieeetr}
\bibliography{browsed}
\end{document}
